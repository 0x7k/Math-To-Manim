\documentclass[10pt]{article}
\usepackage[utf8]{inputenc}
\usepackage[T1]{fontenc}
\usepackage{amsmath}
\usepackage{amsfonts}
\usepackage{amssymb}
\usepackage[version=4]{mhchem}
\usepackage{stmaryrd}
\usepackage{bbold}

\title{Benamou-Brenier }

\author{}
\date{}


\begin{document}
\maketitle
March 22, 2016

\section*{1 Introduction of the metric space \(\mathbb{W}_{p}(\Omega)\)}
Let \(\Omega\) be an open set of \(\mathbb{R}^{n}\). Let

\[
\mathcal{P}_{p}(\Omega):=\left\{\left.\mu \in \mathcal{M}(\Omega)\left|\mu(\Omega)=1, \quad \int_{\Omega}\right| x\right|^{p} \mathrm{~d} \mu(x)<+\infty\right\} .
\]

For every \(\mu, \nu \in \mathcal{P}_{p}(\Omega)\) we define the Wasserstein distance between \(\mu\) and \(\nu\) as

\[
W_{p}(\mu, \nu):=\left(\min \left\{\int_{\Omega \times \Omega}|x-y|^{p} \mathrm{~d} \gamma(x, y) \mid \gamma \in \operatorname{ADM}(\mu, \nu)\right\}\right)^{1 / p}
\]

where \(\operatorname{ADM}(\mu, \nu)\) is the class of all admissible transport plan between \(\mu\) and \(\nu\). We can replace \(\left(\Omega,|\cdot|{ }^{p}\right)\) with a complete separable metric space \((X, d)\).\\[0pt]
Proposition 1.1. [AG13, Theorem 2.2], [San15, Lemma 5.4]. \(W_{p}(\cdot, \cdot)\) is a distance on \(\mathcal{P}_{p}(\Omega)\).

In particular we are going to refer to the metric space \(\mathbb{W}_{p}(\Omega)=\left(\mathcal{P}_{p}(\Omega), W_{p}(\cdot, \cdot)\right)\). Let us introduce the following two definitions\\
Definition 1.2. Given a metric space ( \(X, d\) ) we define a curve on \(X\) to be a continuous function \(\omega:[0,1] \rightarrow X\). Given a curve \(\omega\) on \(X\) we define the metric derivative of \(\omega\) to be

\[
\left|\omega^{\prime}\right|(t):=\lim _{s \rightarrow t} \frac{d(\omega(s), \omega(t))}{|t-s|}
\]

whenever the limit exists. We say that a curve \(\omega\) is absolutely continuous if there exists a function \(g \in L^{1}([0,1])\) such that

\[
d(\omega(t), \omega(s)) \leq \int_{s}^{t} g(\tau) \mathrm{d} \tau \quad \text { for all } s, t \in[0,1] \text { with } s<t .
\]

Theorem 1.3. [AGS08, Theorem 1.1.2]. An absolutely continuous curve \(\omega\) on a metric space \(X\) admits the metric derivative almost-everywhere and it holds

\[
d(\omega(t), \omega(s)) \leq \int_{s}^{t}\left|\omega^{\prime}\right|(\tau) \mathrm{d} \tau \quad \text { for all } s, t \in[0,1] \text { with } s<t \text {. }
\]

Definition 1.4. For a curve \(\omega\) on a metric space \(X\) we define

\[
\text { length }(\omega):=\sup \left\{\sum_{i=0}^{n-1} d\left(\omega\left(t_{i}\right), \omega\left(t_{i+1}\right) \mid n \geq 1, \quad 0=t_{0}<t_{1}<\ldots<t_{n}=1\right\}\right.
\]

Proposition 1.5. For each absolutely continuous curve \(\omega\) on \(X\) it holds

\[
\operatorname{length}(\omega)=\int_{0}^{1}\left|\omega^{\prime}\right|(t) \mathrm{d} t
\]

\section*{\(2 \mathbb{W}_{p}(\Omega)\) is a Geodesic space}
A metric space \((X, d)\) is called a geodesic space if for any pair of points \(x, y\) there exists an absolutely continuous curve \(\omega\) such that \(\omega(0)=x, \omega(1)=y\) and

\[
d(x, y)=\operatorname{length}(\omega)=\min \{\operatorname{length}(\gamma) \mid \gamma \text { a. c. curve with } \gamma(0)=x, \gamma(1)=y\}
\]

A curve \(\omega\) attaining the minimum above is called a geodesic.\\
Definition 2.1. A curve \(\omega\) is instead called a constant speed geodesic on \(X\) if it holds

\[
d(\omega(t), \omega(s))=|t-s| d(x, y) \quad \text { for every } t, s \in[0,1]
\]

It is immediate to check that each constant speed geodesic is geodesic.\\[0pt]
Theorem 2.2. [San15, Theorem 5.27]. If \(\Omega\) is convex then for every pair of measure \(\mu, \nu \in \mathbb{W}_{p}(\Omega)\) there exists a constant speed geodesic between \(\mu\) and \(\nu\). As a consequence \(\mathbb{W}_{p}(\Omega)\) is a geodesic space.

Proof. Let \(\mu, \nu \in \mathbb{W}_{p}(\Omega)\) and let \(\gamma \in \operatorname{ADM}(\mu, \nu)\) be an optimal transport plan. For every \(t \in[0,1]\) define the map \(\pi_{t}: \Omega \times \Omega \rightarrow \Omega\) as \(\pi_{t}(x, y)=t x+(1-t) y\) and the curve of measure \(\gamma_{t}:=\left(\pi_{t}\right)_{\#} \gamma\). We notice that \(\gamma_{t}\) is a constant speed geodesic. Indeed, for every \(t>s\) we have \(\lambda_{t}^{s}:=\left(\pi_{t}, \pi_{s}\right)_{\#} \gamma \in \operatorname{ADM}\left(\gamma_{t}, \gamma_{s}\right):\)

\[
\left.\left(\pi_{t}, \pi_{s}\right)_{\#} \gamma(A \times \Omega)=\gamma\left(\pi_{t}, \pi_{s}\right)^{-1}(A \times \Omega)\right)=\gamma\left(\pi_{t}^{-1}(A)\right)=\left(\pi_{t}\right)_{\#} \gamma(A)=\gamma_{t}(A)
\]

and analogously for \(\left(\pi_{t}, \pi_{s}\right)_{\#} \gamma(\Omega \times B)\). Thus

\[
\begin{aligned}
W_{p}\left(\gamma_{t}, \gamma_{s}\right)^{p} & \leq \int_{\Omega \times \Omega}|x-y|^{p} \mathrm{~d} \lambda_{t}^{s}(x, y) \\
& =\int_{\Omega \times \Omega}\left|\pi_{t}(x, y)-\pi_{s}(x, y)\right|^{p} \mathrm{~d} \gamma(x, y) \\
& =\int_{\Omega \times \Omega}|(t-s) x-(t-s) y|^{p} \mathrm{~d} \gamma(x, y) \\
& =(t-s)^{p} \int_{\Omega \times \Omega}|x-y|^{p} \mathrm{~d} \gamma(x, y) \\
& =(t-s)^{p} W_{p}(\mu, \nu)^{p} .
\end{aligned}
\]

Conversely for \(s<t\)

\[
\begin{aligned}
W_{p}(\mu, \nu) & \leq W_{p}\left(\mu, \gamma_{s}\right)+W_{p}\left(\gamma_{t}, \gamma_{s}\right)+W_{p}\left(\gamma_{t}, \nu\right) \\
& \leq(s+1-t) W_{p}(\mu, \nu)+W_{p}\left(\gamma_{t}, \gamma_{s}\right)
\end{aligned}
\]

which implies

\[
W_{p}(\mu, \nu)(t-s) \leq W_{p}\left(\gamma_{t}, \gamma_{s}\right)
\]

\section*{3 Relation between continuity equation and Geodesic}
Given a curve if measure \(\left\{\mu_{t}\right\}_{t \in[0,1]}\) and a vector field \(\mathbf{v}: \Omega \times[0,1]\) (we set \(\left.\mathbf{v}_{t}(x):=\mathbf{v}(x, t)\right)\) such that \(\mathbf{v}_{t} \in L^{p}\left(\Omega ; \mu_{t}\right)\). We say that the pair \(\left(\mathbf{v}_{t}, \mu_{t}\right)\) satisfies the continuity equation

\[
\partial_{t} \mu_{t}+\nabla \cdot\left(\mathbf{v}_{t} \mu_{t}\right)=0
\]

in the weak sense if for every \(\phi \in C_{c}^{\infty}(\Omega \times[0,1])\) it holds

\[
\frac{d}{d t} \int_{\Omega} \phi \mathrm{d} \mu_{t}(x)=\int_{\Omega} \nabla \phi \cdot \mathbf{v}_{t}(x) \mathrm{d} \mu_{t}
\]

Theorem 3.1 (Characterization of the solution of CE). [San15, Theorem 4.4] Let \(\left\{\mu_{t}\right\}_{t \in[0,1]}\) be a curve of measure absolutely continuous wrt to Lebesgue and \(\mathbf{v}_{t} \in L^{1}\left(\Omega ; \mu_{t}\right)\) be a Borel vector field such that the couple \(\left(\mathbf{v}_{t}, \mu_{t}\right)\) solves the continuity equation

\[
\partial \mu_{t}+\nabla \cdot\left(\mu_{t} \mathbf{v}_{t}\right)=0
\]

in the weak sense. Then \(\mu_{t}=\left(Y_{t}\right)_{\#} \mu_{0}\) where

\[
\left\{\begin{array}{l}
\frac{d}{d t} Y_{t}(x)=\mathbf{v}_{t}\left(Y_{t}(x)\right) \\
Y_{0}(x)=x
\end{array}\right.
\]

Theorem 3.2. [San15, Theorem 5.14]. Let \(\left\{\mu_{t}\right\}_{t \in[0,1]}\) be an absolutely continuous curve on \(\mathbb{W}_{p}(\Omega)\) with \(\Omega\) a compact and convex set. Then, for a.e. \(t \in[0,1]\) there exists a Borel vector field \(\mathbf{v}_{t} \in L^{p}\left(\Omega ; \mu_{t}\right)\) such that \(\left\|\mathbf{v}_{t}\right\|_{L^{p}\left(\mu_{t}\right)} \leq\left|\mu^{\prime}\right|(t)\) and such that the pair \(\left(\mathbf{v}_{t}, \mu_{t}\right)\) satisfies the continuity equation in the weak sense.

Sketch of the sketch proof of (3.2). Let \(\mu_{0}\) and \(\mu_{1}\) be absolutely continuous wrt to Lebesgue and let \(T\) be the optimal transport map between \(\mu_{0}\) and \(\mu_{1}\). Assume that \(\mu_{t}:=\left(T_{t}(x)\right)_{\#} \mu_{0}\) where \(T_{t}(x)=t I d+(1-t) T(x)\). Then clearly the vector field \(v_{t}(x)=T(x)-x=\partial_{t} T_{t}(x)\) detect the constant velocity of the particle \(x\) while moving on the segment joining \(x\) and \(T(x)\). In particular if I set \(\mathbf{v}_{t}(y):=\left(v_{t} \circ T_{t}\right)^{-1}(y)\) the vector field \(\mathbf{v}_{t}\) is codifying the velocity of a generic particle \(y \in \Omega\) during the transport process. This is the natural vector field to associate to "segment of absolutely continuous measure in \(\Omega\) ". In particular this is doing exactly what we need, indeed by construction

\[
\begin{aligned}
\frac{d}{d t} \int_{\Omega} \psi \mathrm{d} \mu_{t} & =\frac{d}{d t} \int_{\Omega} \psi \circ T_{t}(x) \mathrm{d} \mu_{0}=\int_{\Omega} \nabla \psi\left(T_{t}(x)\right) \cdot v_{t}(x) \mathrm{d} \mu_{0} \\
& =\int_{\Omega} \nabla \psi(y)\left(v_{t} \circ T_{t}\right)^{-1}(y) \mathrm{d} \mu_{t}(y)=\int_{\Omega} \nabla \psi \cdot \mathbf{v}_{t} \mathrm{~d} \mu_{t}
\end{aligned}
\]

Moreover this vector field has the good property that

\[
\begin{aligned}
\int_{a}^{b}\left\|\mathbf{v}_{t}\right\|_{L}^{p}\left(\mu_{t}\right) \mathrm{d} t & =\int_{a}^{b} \int_{\Omega}\left|\mathbf{v}_{t}\right|^{p} \mathrm{~d} \mu_{t} \mathrm{~d} t \\
& =(b-a) W_{p}\left(\mu_{0}, \mu_{1}\right)=W_{p}\left(\mu_{a}, \mu_{b}\right)=\int_{a}^{b}\left|\mu^{\prime}\right|(t) \mathrm{d} t
\end{aligned}
\]

In particular \(\left\|\mathbf{v}_{t}\right\|_{L^{p}\left(\mu_{t}\right)}=\left|\mu^{\prime}\right|(t)\). With an approximation argument we conclude that this holds for all the curves of measure.

Sketch proof of (3.2). Let us assume for sake of simplicity that the curve \(\mu=\left\{\mu_{t}\right\}_{t \in[0,1]}\) is made by absolutely continuous measures with respect to the Lebesgue measure \(\mathcal{L}^{n}\) and that \(\sup _{t}\left\{\mu_{t}(\Omega)\right\}<+\infty\). We can also assume, up to a reparametrization that \(\mu\) is a Lipschitz curve. In particular, it holds that

\[
\left|\mu^{\prime}\right|(t) \leq \operatorname{Lip}(\mu) \quad \text { for a.e. } t \in[0,1] .
\]

We do our proof by approximation. Let us fix a number \(k \in \mathbb{N}\) and consider a partition of \([0,1]\) into \(\left[\frac{i}{k}, \frac{i+1}{k}\right]\) for \(i=0, \ldots, k-1\). Each measure \(\mu_{i}^{k}:=\mu_{\frac{i}{k}}\) is absolutely continuous and thus it make sense to consider the optimal transport map \(T^{i, k}: \Omega \rightarrow \Omega\) such that \(\left(T^{i, k}\right)_{\#} \mu_{i}^{k}=\mu_{i+1}^{k}\). The interpolation

\[
T_{t}^{i, k}(x):=(i+1-k t) I d+(k t-i) T^{i, k}
\]

detect the position of the particle \(x\) at an intermediate time \(t \in\left[\frac{i}{k}, \frac{i+1}{k}\right]\). We thus define the curve of measures given by

\[
\mu_{t}^{k}:=\left(T_{t}^{i, k}\right)_{\#} \mu_{i}^{k} \quad \text { for } t \in\left[\frac{i}{k}, \frac{i+1}{k}\right]
\]

which means that the particle located at \(x\) at time \(\frac{i}{k}\) goes in the position \(T^{i, k}(x)\) at time \(\frac{i+1}{k}\) by moving at constant speed along the segment joining \(x\) and \(T^{i, k}(x)\). In particular the vector field \(\mathbf{v}^{i, k}(x):=k\left(T^{i, k}(x)-x\right)\) detect the constant velocity of the particle \(x\) in the time interval considered. Moreover since the map \(T_{t}^{i, k}\) is injective ([San15, Lemma 4.23] ) we can always define the global vector field

\[
\mathbf{v}_{t}^{k}(y):=\mathbf{v}^{i, k} \circ\left(T_{t}^{i, k}\right)^{-1}(y), \quad \text { for } t \in\left[\frac{i}{k}, \frac{i+1}{k}\right]
\]

We now provide two key estimates on \(\mathbf{v}_{t}^{k}\) due to (3.4) and to the construction performed. For the first one we have

\[
\begin{aligned}
\left\|\mathbf{v}_{t}^{k}\right\|_{L^{p}\left(\mu_{t}^{k}\right)}^{p} & =\int_{\Omega}\left|\mathbf{v}_{t}^{k}\right|^{p} \mathrm{~d} \mu_{t}^{k}=\int_{\Omega}\left|\mathbf{v}^{i, k}\right|^{p} \mathrm{~d} \mu_{i}^{k}=\int_{\Omega}\left|\mathbf{v}^{i, k}\right|^{p} \mathrm{~d} \mu_{\frac{i}{k}} \\
& =k^{p} \int_{\Omega}\left|T^{i, k}(x)-x\right|^{p} \mathrm{~d} \mu_{\frac{i}{k}}=k^{p} W_{p}\left(\mu_{\frac{i}{k}}, \mu_{\frac{i+1}{k}}\right)^{p} \\
& \leq k^{p}\left(\int_{\frac{i}{k}}^{\frac{i+1}{k}}\left|\mu^{\prime}\right|(t) \mathrm{d} t\right)^{p} \leq k \int_{\frac{i}{k}}^{\frac{i+1}{k}}\left|\mu^{\prime}\right|(t)^{p} \mathrm{~d} t \leq \operatorname{Lip}(\mu)^{p}
\end{aligned}
\]

Moreover, for every \(a<b \in[0,1]\) with \(\frac{i_{a}}{k} \leq a \leq \frac{i_{a}+1}{k}, \frac{i_{b}}{k} \leq a \leq \frac{i_{b}+1}{k}\) we have also

\[
\begin{aligned}
\int_{a}^{b}\left\|\mathbf{v}_{t}^{k}\right\|_{L^{p}\left(\mu_{t}^{k}\right)}^{p} \mathrm{~d} t & \leq \sum_{i=i_{a}}^{i_{b}} \int_{\frac{i}{k}}^{\frac{i+1}{k}} \int_{\Omega}\left|\mathbf{v}_{i, k}\right|^{p} \mathrm{~d} \mu_{i}^{k} \mathrm{~d} t \\
& \leq k^{p-1} \sum_{i=i_{a}}^{i_{b}} W_{p}\left(\mu_{\frac{i}{k}}, \mu_{\frac{i+1}{k}}\right)^{p} \leq k^{p-1} \sum_{i=i_{a}}^{i_{b}}\left(\int_{\frac{i}{k}}^{\frac{i+1}{k}}\left|\mu^{\prime}\right|(t) \mathrm{d} t\right)^{p} \\
\text { Jensen'sinequality }^{k} & \leq \sum_{i=i_{a}}^{i_{b}} \int_{\frac{i}{k}}^{\frac{i+1}{k}}\left|\mu^{\prime}\right|(t)^{p} \mathrm{~d} t=\int_{\frac{i_{a}}{k}}^{\frac{i_{b}+1}{k}}\left|\mu^{\prime}\right|(t)^{p} \mathrm{~d} t \\
& \leq \int_{a}^{b}\left|\mu^{\prime}\right|(t)^{p} \mathrm{~d} t+\frac{2 \operatorname{Lip}(\mu)^{p}}{k}
\end{aligned}
\]

Let us check that \(\left(\mathbf{v}_{t}^{k}, \mu_{t}^{k}\right)\) solves the continuity equation in a weak sense. Let \(\phi \in C_{c}^{\infty}(\Omega \times\) \([0,1])\) :

\[
\begin{aligned}
\frac{d}{d t} \int_{\Omega} \phi \mathrm{d} \mu_{t}^{k} \mathrm{~d} t & =\sum_{i=0}^{k-1} \frac{d}{d t} \int_{\Omega} \phi \mathrm{d} \mu_{t}^{i, k}=\sum_{i=0}^{k-1} \frac{d}{d t} \int_{\Omega} \phi \circ T_{t}^{i, k} \mathrm{~d} \mu_{i}^{k} \\
& =\sum_{i=0}^{k-1} \int_{\Omega} \nabla \phi \cdot \mathbf{v}^{i, k} \mathrm{~d} \mu_{i}^{k}=\int_{\Omega} \nabla \phi \cdot \mathbf{v}_{t}^{k} \mathrm{~d} \mu_{t}^{k} .
\end{aligned}
\]

We now define the curve of vector-valued Radon measure \(E_{t}^{k}:=\mathbf{v}_{t}^{k} \mu_{t}^{k}\) on \(\Omega\) for \(t \in[0,1]\) and we notice that, thanks to (3.6), it holds:

\[
\left|E_{t}^{k}\right|(\Omega)=\int_{\Omega}\left|\mathbf{v}_{t}^{k}\right|_{L^{1}\left(\mu_{t}^{k}\right)} \leq C(\Omega)\left\|\mathbf{v}_{t}^{k}\right\|_{L^{p}\left(\mu_{t}^{k}\right)} \leq C(\Omega) \operatorname{Lip}(\mu) .
\]

In particular, since \(\Omega\) is compact, for every \(t\) we can find a vector-valued Radon measure \(E_{t}\) such that \(E_{t}^{k} \rightharpoonup^{*} E_{t}\). On the other hand we have that \(\mu_{t}^{k} \rightharpoonup^{*} \mu_{t}\). We want to show that \(E_{t} \ll \mu_{t}\) for every \(t \in[0,1]\). Let \(B_{r_{0}} \subset \subset \Omega\) be such that \(\mu_{t}\left(B_{r_{0}}\right)=0\) and observe that for almost every \(r<r_{0}\) we have that

\[
\begin{aligned}
\left|E_{t}\right|\left(B_{r}\right) & \leq \liminf _{k \rightarrow+\infty}\left|E_{t}^{k}\right|\left(B_{r}\right) \leq \liminf _{k \rightarrow+\infty} \int_{B_{r}}\left|\mathbf{v}_{t}^{k}\right| \mathrm{d} \mu_{t}^{k} \\
& \leq \liminf _{k \rightarrow+\infty}^{k} \mu_{t}^{k}\left(B_{r}\right)^{1-1 / p}\left(\int_{B_{r}}\left|\mathbf{v}_{t}^{k}\right|^{p} \mathrm{~d} \mu_{t}^{k}\right)^{\frac{1}{p}} \\
& \leq \operatorname{Lip}(\mu) \mu_{t}\left(B_{r}\right)^{1-1 / p}=0
\end{aligned}
\]

where we have exploited again (3.6). In particular \(E_{t}=\mathbf{v}_{t} \mu_{t}\) for some vector field \(\mathbf{v}_{t} \in L^{1}\left(\Omega ; \mu_{t}\right)\). By expoiting the semiconitnuity of a particular functional (the so-called Benamou-Brenier functional, which is basically the relaxed version of the functional minimized in the Benamou-Brenier Theorem below, see [San15, Proposition 5.18]) we also reach, for every \(0 \leq a<b \leq 1\)

\[
\int_{a}^{b}\left\|\mathbf{v}_{t}\right\|_{L^{p}\left(\mu_{t}\right)}^{p} \mathrm{~d} t \leq \underset{k}{\lim \inf } \int_{a}^{b}\left\|\mathbf{v}_{t}^{k}\right\|_{L^{p}\left(\mu_{t}^{k}\right)}^{p} \mathrm{~d} t
\]

which implies \(\left\|\mathbf{v}_{t}\right\|_{L^{p}\left(\mu_{t}\right)}^{p}<+\infty\) for a.e. \(t \in[0,1]\) and, together with (3.7), leads us to

\[
\int_{a}^{b}\left\|\mathbf{v}_{t}\right\|_{L^{p}\left(\mu_{t}\right)}^{p} \mathrm{~d} t \leq \liminf _{k} \int_{a}^{b}\left\|\mathbf{v}_{t}^{k}\right\|_{L^{p}\left(\mu_{t}^{k}\right)}^{p} \mathrm{~d} t \leq \int_{a}^{b}\left|\mu^{\prime}\right|(t)^{p} \mathrm{~d} t .
\]

Thanks to the arbitrariness of \(a, b \in[0,1]\) we conclude \(\left\|\mathbf{v}_{t}\right\|_{L^{p}\left(\mu_{t}\right)} \leq\left|\mu^{\prime}\right|(t)\) for a.e. \(t \in[0,1]\). We finally need to check that the pair \(\left(\mathbf{v}_{t}, \mu_{t}\right)\) satisfies the continuity equation in the week sense. Let \(\phi \in C_{c}^{\infty}(\Omega)\) and \(a \in C_{c}^{\infty}([0,1])\). The function

\[
f(t):=a^{\prime}(t) \int_{\Omega} \phi \mathrm{d} \mu_{t}, \quad g(t)=a(t) \int_{\Omega} \nabla \phi \cdot d E_{t}
\]

are in \(L^{1}([0,1])\) and are the point-wise limit a. e. of

\[
f_{k}(t)=a^{\prime}(t) \int_{\Omega} \phi \mathrm{d} \mu_{t}^{k}, \quad g_{k}(t)=a(t) \int_{\Omega} \nabla \phi \cdot d E_{t}^{k} .
\]

Thus by the dominated convergence theorem we have

\[
\begin{aligned}
\int_{0}^{1} a^{\prime}(t) \int_{\Omega} \phi \mathrm{d} \mu_{t} \mathrm{~d} t & =\lim _{k} \int_{0}^{1} a^{\prime}(t) \int_{\Omega} \phi \mathrm{d} \mu_{t}^{k} \mathrm{~d} t \\
& =-\lim _{k} \int_{0}^{1} a(t) \frac{d}{d t} \int_{\Omega} \phi \mathrm{d} \mu_{t}^{k} \mathrm{~d} t \\
& =-\lim _{k} \int_{0}^{1} a(t) \int_{\Omega} \nabla \phi \cdot \mathbf{v}_{t}^{k} d \mu_{t}^{k} \mathrm{~d} t \\
& =-\lim _{k} \int_{0}^{1} a(t) \int_{\Omega} \nabla \phi \cdot d E_{t}^{k} \mathrm{~d} t \\
& =-\int_{0}^{1} a(t) \int_{\Omega} \nabla \phi \cdot d E_{t} \mathrm{~d} t \\
& =-\int_{0}^{1} a(t) \int_{\Omega} \nabla \phi \cdot \mathbf{v}_{t} \mathrm{~d} \mu_{t} \mathrm{~d} t
\end{aligned}
\]

In particular this means that

\[
\frac{d}{d t} \int_{\Omega} \phi \mathrm{d} \mu_{t}=\int_{\Omega} \nabla \phi \cdot \mathbf{v}_{t} \mathrm{~d} \mu_{t}
\]

Theorem 3.3. [San15, Theorem 5.14]. Let \(\left\{\mu_{t}\right\}_{t \in[0,1]}\) be a curve of measure and \(\mathbf{v}_{t} \in\) \(L^{p}\left(\Omega ; \mu_{t}\right)\) be a Borel vector field such that \(\int_{0}^{1}\left\|\mathbf{v}_{t}\right\|_{L^{p}\left(\mu_{t}\right)} \mathrm{d} t<+\infty\). Assume that the pair \(\left(\mathbf{v}_{t}, \mu_{t}\right)\) satisfies the continuity equation. Then, up to redefine \(t \mapsto \mu_{t}\) on an \(\mathcal{L}^{1}\)-negligible set of time, \(\mu_{t}\) is an absolutely continuous curve on \(\mathbb{W}_{p}(\Omega)\) and \(\left|\mu^{\prime}\right|(t) \leq\left\|\mathbf{v}_{t}\right\|_{L^{p}\left(\mu_{t}\right)}\) for almost every \(t \in[0,1]\).

Sketch proof of 3.3. Let us assume that the curve \(\mu_{t} \ll \mathcal{L}^{n}\) for all \(t\) (the complete proof need a suitable argument of regularization that can be provided, we refer the reader to [San15]). Then, thanks to Theorem 3.1 we have that \(\mu_{t}=\left(Y_{t}\right)_{\#} \mu_{0}\) where \(Y_{t}\) is a vector field satisfying

\[
\left\{\begin{array}{l}
\frac{d}{d Y_{t}} Y_{t}(x)=\mathbf{v}_{t}\left(Y_{t}(x)\right) ; \\
Y_{0}(x)=x
\end{array}\right.
\]

If we now fix \(t \in(0,1)\) and \(h\) small enough so that \(t+h \in(0,1)\) we have that the plan \(\gamma=\left(Y_{t}, Y_{t+h}\right)_{\#} \mu_{0}\) is a transport plan between \(\mu_{t}\) and \(\mu_{t+h}\). In particular

\[
\begin{aligned}
W_{p}\left(\mu_{t}, \mu_{t+h}\right)^{p} & \leq \int_{\Omega \times \Omega}|x-y|^{p} \mathrm{~d} \gamma(x, y) \\
& =\int_{\Omega}\left|Y_{t}(x)-Y_{t+h}(x)\right|^{p} \mathrm{~d} \mu_{0}(x) \\
& =\int_{\Omega}\left|\int_{t}^{t+h} \frac{d}{d s} Y_{s}(x) \mathrm{d} s\right|^{p} \mathrm{~d} \mu_{0}(x) \\
& \leq|h|^{\frac{p}{q}} \int_{\Omega} \int_{t}^{t+h}\left|\frac{d}{d s} Y_{s}(x)\right|^{p} \mathrm{~d} s \mathrm{~d} \mu_{0}(x) \\
& =|h|^{\frac{p}{q}} \int_{\Omega} \int_{t}^{t+h}\left|\mathbf{v}_{s}\left(Y_{s}(x)\right)\right|^{p} \mathrm{~d} s \mathrm{~d} \mu_{0}(x) \\
& =|h|^{\frac{p}{q}} \int_{t}^{t+h} \int_{\Omega}\left|\mathbf{v}_{s}(y)\right|^{p} \mathrm{~d} \mu_{t}(y) \mathrm{d} s .
\end{aligned}
\]

Hence

\[
\frac{W_{p}\left(\mu_{t}, \mu_{t+h}\right)}{|h|} \leq\left(\frac{1}{|h|} \int_{t}^{t+h}\left\|\mathbf{v}_{s}(y)\right\|_{L^{p}\left(\mu_{s}\right)}^{p} \mathrm{~d} s\right)^{\frac{1}{p}}
\]

and by sending \(h\) to zero we achieve, at every \(t \in[0,1]\) which is a Lebesgue point of \(\left\|\mathbf{v}_{t}(y)\right\|_{L^{p}\left(\mu_{t}\right)}^{p}:\)

\[
\left|\mu^{\prime}\right|(t) \leq\left\|\mathbf{v}_{t}(y)\right\|_{L^{p}\left(\mu_{t}\right)} .
\]

\section*{4 The Benamou-Brenier Formula}
Theorem 4.1. [BB00, Proposition 1.1] For every \(\mu, \nu \in \mathbb{W}_{p}(\Omega)\) it holds

\[
\begin{aligned}
W_{p}(\mu, \nu)^{p} & =\min \left\{\int_{0}^{1}\left\|\mathbf{v}_{t}\right\|_{L^{p}\left(\mu_{t}\right)}^{p} \mathrm{~d} t \mid\left(\mathbf{v}_{t}, \mu_{t}\right) \text { satisfies (3.1) and } \mu_{0}=\mu, \mu_{1}=\nu\right\} \\
& =\min \left\{\int_{0}^{1} \int_{\Omega}\left|\mathbf{v}_{t}\right|^{p} \mathrm{~d} \mu_{t} \mathrm{~d} t \mid\left(\mathbf{v}_{t}, \mu_{t}\right) \text { satisfies (3.1) and } \mu_{0}=\mu, \mu_{1}=\nu\right\} .
\end{aligned}
\]

Proof. Let us set, for the sake of brevity (and thus of clarity)

\[
\sigma=\inf \left\{\int_{0}^{1}\left\|\mathbf{v}_{t}\right\|_{L^{p}\left(\mu_{t}\right)}^{p} \mathrm{~d} t \mid\left(\mathbf{v}_{t}, \mu_{t}\right) \text { satisfies (3.1) and } \mu_{0}=\mu, \mu_{1}=\nu\right\} .
\]

Let \(\left(\mathbf{v}_{t}, \mu_{t}\right)\) be a solution to (3.1) satisfying \(\mu_{0}=\mu, \mu_{1}=\nu\) and such that \(\int_{0}^{1}\left\|\mathbf{v}_{t}\right\|_{L^{p}\left(\mu_{t}\right)} \mathrm{d} t<\) \(+\infty\). Thanks to Theorem 3.3 we can conclude that the curve \(\mu_{t}\) is an absolutely continuous curve and \(\left|\mu^{\prime}\right|(t) \leq\left\|\mathbf{v}_{t}\right\|_{L^{p}\left(\mu_{t}\right)}\). In particular, thanks to 1.3 we have

\[
W_{p}(\mu, \nu)^{p} \leq\left(\int_{0}^{1}\left|\mu^{\prime}\right|(t) \mathrm{d} t\right)^{p} \leq \int_{0}^{1}\left\|\mathbf{v}_{t}\right\|_{L^{p}\left(\mu_{t}\right)}^{p} \mathrm{~d} t
\]

and thus

\[
W_{p}(\mu, \nu)^{p} \leq \sigma .
\]

Conversely, let \(\left\{\mu_{t}\right\}_{t \in[0,1]}\) be a constant speed geodesic connecting \(\mu\) and \(\nu\). Then, according to Theorem 3.2 for a.e. \(t \in[0,1]\) we can find a Borel vector field \(\mathbf{v}_{t} \in L^{p}\left(\Omega ; \mu_{t}\right)\) such that \(\left\|\mathbf{v}_{t}\right\|_{L^{p}\left(\mu_{t}\right)} \leq\left|\mu^{\prime}\right|(t)\) and the pair \(\left(\mathbf{v}_{t}, \mu_{t}\right)\) satisfies the weak equation. Thanks to Theorem 3.2 since \(\left(\mathbf{v}_{t}, \mu_{t}\right)\) satisfies (3.1) we have also \(\left\|\mathbf{v}_{t}\right\|_{L^{p}\left(\mu_{t}\right)} \geq\left|\mu^{\prime}\right|(t)\) and thus \(\left\|\mathbf{v}_{t}\right\|_{L^{p}\left(\mu_{t}\right)}=\) \(\left|\mu^{\prime}\right|(t)\). Since \(\mu_{t}\) is a constant speed geodesic we have \(\left|\mu^{\prime}\right|(t)=W_{p}(\mu, \nu)\) (it is an easy computation) and thus

\[
\sigma \geq W_{p}(\mu, \nu)^{p}=\int_{0}^{1}\left|\mu^{\prime}\right|(t)^{p} \mathrm{~d} t=\int_{0}^{1}\left\|\mathbf{v}_{t}\right\|_{L^{p}\left(\mu_{t}\right)}^{p} \mathrm{~d} t \geq \sigma
\]

which implies \(W_{p}(\mu, \nu)=\sigma\) and that the minimum is attained by \(\left(\mathbf{v}_{t}, \mu_{t}\right)\).

\section*{5 The Tangent space to \(\mathbb{W}_{p}(\Omega)\)}
Given a an absolutely continuous curve of measure \(\mu=\left\{\mu_{t}\right\}_{t \in[0,1]}\) we have shown in Theorem 3.2 that we can always associate to \(\mu_{t}\) a Borel vector field \(\mathbf{v}_{t}\) so that the pair \(\left(\mathbf{v}_{t}, \mu_{t}\right)\) satisfies the continuity equation (3.1). In particular we have seen that this vector field plays the role of the "tangent vector field to the measure \(\mu\) " and among all the possible vector field, that coupled with \(\mu_{t}\) satisfy (3.1) is the ones with minimal \(\left\|\mathbf{v}_{t}\right\|_{L^{p}\left(\mu_{t}\right)}\). In particular, for every \(t \in[0,1]\) we have that

\[
\left\|\mathbf{v}_{t}\right\|_{L^{p}\left(\mu_{t}\right)}=\min \left\{\left.\left(\int_{\Omega}\left|\mathbf{w}_{t}\right|^{p} \mathrm{~d} \mu_{t}\right)^{\frac{1}{p}} \right\rvert\, \mathbf{w}_{t} \in L^{p}\left(\Omega ; \mu_{t}\right), \partial_{t} \mu_{t}+\nabla \cdot\left(\mathbf{w}_{t} \mu_{t}\right)=0\right\} .
\]

This leads us to the following intuition. Set

\[
\mathcal{M}:=\left\{\rho: \mathbb{R}^{n} \rightarrow \mathbb{R}^{+}, \int_{\mathbb{R}^{n}} \rho \mathrm{~d} x=1\right\}
\]

The Wasserstein distance \(W_{p}\) gives us a sort of Riemaniann structure on \(\mathcal{M}\). Assume for the sake of simplicity that \(p=2\) and choose a \(\rho \in C^{1}\left(\mathbb{R}^{n}\right)\). If we pick up \(s \in T_{\rho} \mathcal{M}\), based on the intuition above, we can reread [Ott01] the tangent space by looking for a vector field \(\mathbf{v} \in L^{2}(\Omega ; \rho \mathrm{d} x)\) such that

\[
\|\mathbf{v}\|_{L^{2}(\rho \mathrm{~d} x)}=\min \left\{\left.\left(\int_{\Omega}|\mathbf{w}|^{2} \rho \mathrm{~d} x\right)^{\frac{1}{2}} \right\rvert\, \mathbf{w} \in L^{2}(\Omega ; \rho \mathrm{d} x), s+\nabla \cdot(\mathbf{w} \rho)=0\right\} .
\]

If we consider the first variation of the energy at \(\mathbf{v}\) among variations \(\mathbf{u} \in C_{c}^{\infty}\left(\Omega ; \mathbb{R}^{n}\right)\) such that \(\nabla \cdot(\mathbf{u} \rho)=0\) something interesting shows up:

\[
0=\left.\frac{d}{d \varepsilon}\right|_{\varepsilon=0} \int_{\Omega}|\mathbf{v}+\varepsilon \mathbf{u}|^{2} \rho \mathrm{~d} x=\int_{\Omega} 2(\mathbf{v} \cdot \mathbf{u}) \rho \mathrm{d} x .
\]

Heuristically any variation with \(\nabla \cdot(\mathbf{u} \rho)=0\) can be obtained by choosing

\[
\mathbf{u}=\left[\frac{\nabla \times(\mathbf{z} \rho)}{\rho}\right] \mathbb{1}_{\rho \neq 0}
\]

with \(\mathbf{z} \in C_{c}^{\infty}\left(\Omega ; \mathbb{R}^{n}\right)\) and hence, by integrating by parts

\[
\begin{aligned}
0 & =\int_{\Omega}(\mathbf{v} \cdot \mathbf{u}) \rho \mathrm{d} x \\
& =\int_{\rho \neq 0} \mathbf{v} \cdot(\nabla \times(\mathbf{z} \rho)) \mathrm{d} x \\
& =\int_{\Omega} \mathbf{z} \cdot(\nabla \times \mathbf{v}) \rho \mathrm{d} x \quad \text { for any } \mathbf{z} \in C_{c}^{\infty}\left(\Omega ; \mathbb{R}^{n}\right)
\end{aligned}
\]

In particular \(\nabla \times \mathbf{v}=0\) for \(\mu=\rho \mathrm{d} \mathcal{L}^{n}\) a.e. \(x \in \Omega\) and thus up to a \(\mu\) negligible set we must have that \(\mathbf{v}=\nabla p\) for some scalar function \(p: \Omega \rightarrow \mathbb{R}\). In particular this gives us a way to define a sort of weighted \(H^{-1}\) scalar product for \(s_{1}, s_{2} \in T_{\rho} \mathcal{M}\) as

\[
\left\langle s_{1}, s_{2}\right\rangle:=\int_{\mathbb{R}^{n}}\left(\nabla p_{1} \cdot \nabla p_{2}\right) \rho \mathrm{d} x
\]

for \(p_{i}\) satisfying

\[
s_{i}=-\nabla \cdot\left(\rho \nabla p_{i}\right)
\]

(we need to add a boundary condition in order to have unique solution) which is giving us the norm of \(s\) as

\[
\|s\|_{T_{\rho} \mathcal{M}}=\int_{\mathbb{R}^{n}}|\nabla p|^{2} \rho \mathrm{~d} x=\int_{\mathbb{R}^{n}}|\mathbf{v}|^{2} \rho \mathrm{~d} x
\]

\section*{References}
[AG13] Luigi Ambrosio and Nicola Gigli. A user's guide to optimal transport. In Modelling and optimisation of flows on networks, pages 1-155. Springer, 2013.\\[0pt]
[AGS08] Luigi Ambrosio, Nicola Gigli, and Giuseppe Savaré. Gradient flows: in metric spaces and in the space of probability measures. Springer Science \& Business Media, 2008.\\[0pt]
[BB00] Jean-David Benamou and Yann Brenier. A computational fluid mechanics solution to the monge-kantorovich mass transfer problem. Numerische Mathematik, 84(3):375-393, 2000.\\[0pt]
[Ott01] Felix Otto. The geometry of dissipative evolution equations: the porous medium equation. 2001.\\[0pt]
[San15] Filippo Santambrogio. Optimal transport for applied mathematicians. Birkäuser, NY (due in September 2015), 2015.


\end{document}